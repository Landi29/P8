%% bare_jrnl.tex
%% V1.4b
%% 2015/08/26
%% by Michael Shell
%% see http://www.michaelshell.org/
%% for current contact information.
%%
%% This is a skeleton file demonstrating the use of IEEEtran.cls
%% (requires IEEEtran.cls version 1.8b or later) with an IEEE
%% journal paper.
%%
%% Support sites:
%% http://www.michaelshell.org/tex/ieeetran/
%% http://www.ctan.org/pkg/ieeetran
%% and
%% http://www.ieee.org/

%%*************************************************************************
%% Legal Notice:
%% This code is offered as-is without any warranty either expressed or
%% implied; without even the implied warranty of MERCHANTABILITY or
%% FITNESS FOR A PARTICULAR PURPOSE!
%% User assumes all risk.
%% In no event shall the IEEE or any contributor to this code be liable for
%% any damages or losses, including, but not limited to, incidental,
%% consequential, or any other damages, resulting from the use or misuse
%% of any information contained here.
%%
%% All comments are the opinions of their respective authors and are not
%% necessarily endorsed by the IEEE.
%%
%% This work is distributed under the LaTeX Project Public License (LPPL)
%% ( http://www.latex-project.org/ ) version 1.3, and may be freely used,
%% distributed and modified. A copy of the LPPL, version 1.3, is included
%% in the base LaTeX documentation of all distributions of LaTeX released
%% 2003/12/01 or later.
%% Retain all contribution notices and credits.
%% ** Modified files should be clearly indicated as such, including  **
%% ** renaming them and changing author support contact information. **
%%*************************************************************************


% *** Authors should verify (and, if needed, correct) their LaTeX system  ***
% *** with the testflow diagnostic prior to trusting their LaTeX platform ***
% *** with production work. The IEEE's font choices and paper sizes can   ***
% *** trigger bugs that do not appear when using other class files.       ***                          ***
% The testflow support page is at:
% http://www.michaelshell.org/tex/testflow/



\documentclass[journal]{IEEEtran}
%
% If IEEEtran.cls has not been installed into the LaTeX system files,
% manually specify the path to it like:
% \documentclass[journal]{../sty/IEEEtran}

\usepackage{titlesec}

\titleformat{\section}
  {\normalfont\fontsize{12}{17}\sffamily\bfseries}
  {\thesection}
  {1em}
  {}

\titleformat{\subsection}
{\normalfont\fontsize{10}{17}\sffamily}
{\thesubsection}
{1em}
{}

\makeatletter
\def\endthebibliography{%
	\def\@noitemerr{\@latex@warning{Empty `thebibliography' environment}}%
	\endlist
}
\makeatother


\setlength{\parindent}{0cm}


% Some very useful LaTeX packages include:
% (uncomment the ones you want to load)
\usepackage{float}


% *** MISC UTILITY PACKAGES ***
%
%\usepackage{ifpdf}
% Heiko Oberdiek's ifpdf.sty is very useful if you need conditional
% compilation based on whether the output is pdf or dvi.
% usage:
% \ifpdf
%   % pdf code
% \else
%   % dvi code
% \fi
% The latest version of ifpdf.sty can be obtained from:
% http://www.ctan.org/pkg/ifpdf
% Also, note that IEEEtran.cls V1.7 and later provides a builtin
% \ifCLASSINFOpdf conditional that works the same way.
% When switching from latex to pdflatex and vice-versa, the compiler may
% have to be run twice to clear warning/error messages.






% *** CITATION PACKAGES ***
%
%\usepackage{cite}
% cite.sty was written by Donald Arseneau
% V1.6 and later of IEEEtran pre-defines the format of the cite.sty package
% \cite{} output to follow that of the IEEE. Loading the cite package will
% result in citation numbers being automatically sorted and properly
% "compressed/ranged". e.g., [1], [9], [2], [7], [5], [6] without using
% cite.sty will become [1], [2], [5]--[7], [9] using cite.sty. cite.sty's
% \cite will automatically add leading space, if needed. Use cite.sty's
% noadjust option (cite.sty V3.8 and later) if you want to turn this off
% such as if a citation ever needs to be enclosed in parenthesis.
% cite.sty is already installed on most LaTeX systems. Be sure and use
% version 5.0 (2009-03-20) and later if using hyperref.sty.
% The latest version can be obtained at:
% http://www.ctan.org/pkg/cite
% The documentation is contained in the cite.sty file itself.






% *** GRAPHICS RELATED PACKAGES ***
%
\ifCLASSINFOpdf
  \usepackage[pdftex]{graphicx}
  \usepackage{tikz}
  \usepackage{pgfplots}
  \usepackage{adjustbox}
  %\usepackage{subcaption}
  \usepackage{subfig}
  \usetikzlibrary{
  %arrows.meta
  }
  % declare the path(s) where your graphic files are
  % \graphicspath{Graphics/} (Im not sure why this can be commented out)
  % and their extensions so you won't have to specify these with
  % every instance of \includegraphics
  % \DeclareGraphicsExtensions{.pdf,.jpeg,.png}
\else
  % or other class option (dvipsone, dvipdf, if not using dvips). graphicx
  % will default to the driver specified in the system graphics.cfg if no
  % driver is specified.
  % \usepackage[dvips]{graphicx}
  % declare the path(s) where your graphic files are
  % \graphicspath{{../eps/}}
  % and their extensions so you won't have to specify these with
  % every instance of \includegraphics
  % \DeclareGraphicsExtensions{.eps}
\fi
% graphicx was written by David Carlisle and Sebastian Rahtz. It is
% required if you want graphics, photos, etc. graphicx.sty is already
% installed on most LaTeX systems. The latest version and documentation
% can be obtained at:
% http://www.ctan.org/pkg/graphicx
% Another good source of documentation is "Using Imported Graphics in
% LaTeX2e" by Keith Reckdahl which can be found at:
% http://www.ctan.org/pkg/epslatex
%
% latex, and pdflatex in dvi mode, support graphics in encapsulated
% postscript (.eps) format. pdflatex in pdf mode supports graphics
% in .pdf, .jpeg, .png and .mps (metapost) formats. Users should ensure
% that all non-photo figures use a vector format (.eps, .pdf, .mps) and
% not a bitmapped formats (.jpeg, .png). The IEEE frowns on bitmapped formats
% which can result in "jaggedy"/blurry rendering of lines and letters as
% well as large increases in file sizes.
%
% You can find documentation about the pdfTeX application at:
% http://www.tug.org/applications/pdftex





% *** MATH PACKAGES ***
%
\usepackage[utf8]{inputenc}
\usepackage[english]{babel}
\usepackage{amssymb}
\usepackage{amsmath}
\usepackage{amsthm}
\usepackage{amsmath}
\usepackage{todonotes}
\usepackage{hyperref}
\usepackage{tikz}
\usepackage{verbatim}
\usepackage{pgfplots}
\usetikzlibrary{calc,positioning,shapes.geometric,shapes.symbols,shapes.misc,trees}
\usepackage{amsmath}
\newtheorem{theorem}{Theorem}
\newtheorem{definition}{Definition}[section]
\def\definitionautorefname{Definition}
%\usepackage{amsmath}
% A popular package from the American Mathematical Society that provides
% many useful and powerful commands for dealing with mathematics.
%
% Note that the amsmath package sets \interdisplaylinepenalty to 10000
% thus preventing page breaks from occurring within multiline equations. Use:
%\interdisplaylinepenalty=2500
% after loading amsmath to restore such page breaks as IEEEtran.cls normally
% does. amsmath.sty is already installed on most LaTeX systems. The latest
% version and documentation can be obtained at:
% http://www.ctan.org/pkg/amsmath





% *** SPECIALIZED LIST PACKAGES ***
\usepackage[T1]{fontenc}
\usepackage{listings}
\makeatletter
\newcommand\BeraMonottfamily{%
  \def\fvm@Scale{0.85}% scales the font down
  \fontfamily{fvm}\selectfont% selects the Bera Mono font
}
\makeatother

\lstset{
  basicstyle=\BeraMonottfamily,
  frame=single,
  numbers=left,
  numberstyle=\tiny
}
%
%\usepackage{algorithmic}
% algorithmic.sty was written by Peter Williams and Rogerio Brito.
% This package provides an algorithmic environment fo describing algorithms.
% You can use the algorithmic environment in-text or within a figure
% environment to provide for a floating algorithm. Do NOT use the algorithm
% floating environment provided by algorithm.sty (by the same authors) or
% algorithm2e.sty (by Christophe Fiorio) as the IEEE does not use dedicated
% algorithm float types and packages that provide these will not provide
% correct IEEE style captions. The latest version and documentation of
% algorithmic.sty can be obtained at:
% http://www.ctan.org/pkg/algorithms
% Also of interest may be the (relatively newer and more customizable)
% algorithmicx.sty package by Szasz Janos:
% http://www.ctan.org/pkg/algorithmicx




% *** ALIGNMENT PACKAGES ***
%
%\usepackage{array}
% Frank Mittelbach's and David Carlisle's array.sty patches and improves
% the standard LaTeX2e array and tabular environments to provide better
% appearance and additional user controls. As the default LaTeX2e table
% generation code is lacking to the point of almost being broken with
% respect to the quality of the end results, all users are strongly
% advised to use an enhanced (at the very least that provided by array.sty)
% set of table tools. array.sty is already installed on most systems. The
% latest version and documentation can be obtained at:
% http://www.ctan.org/pkg/array


% IEEEtran contains the IEEEeqnarray family of commands that can be used to
% generate multiline equations as well as matrices, tables, etc., of high
% quality.




% *** SUBFIGURE PACKAGES ***
%\ifCLASSOPTIONcompsoc
%  \usepackage[caption=false,font=normalsize,labelfont=sf,textfont=sf]{subfig}
%\else
%  \usepackage[caption=false,font=footnotesize]{subfig}
%\fi
% subfig.sty, written by Steven Douglas Cochran, is the modern replacement
% for subfigure.sty, the latter of which is no longer maintained and is
% incompatible with some LaTeX packages including fixltx2e. However,
% subfig.sty requires and automatically loads Axel Sommerfeldt's caption.sty
% which will override IEEEtran.cls' handling of captions and this will result
% in non-IEEE style figure/table captions. To prevent this problem, be sure
% and invoke subfig.sty's "caption=false" package option (available since
% subfig.sty version 1.3, 2005/06/28) as this is will preserve IEEEtran.cls
% handling of captions.
% Note that the Computer Society format requires a larger sans serif font
% than the serif footnote size font used in traditional IEEE formatting
% and thus the need to invoke different subfig.sty package options depending
% on whether compsoc mode has been enabled.
%
% The latest version and documentation of subfig.sty can be obtained at:
% http://www.ctan.org/pkg/subfig




% *** FLOAT PACKAGES ***
%
%\usepackage{fixltx2e}
% fixltx2e, the successor to the earlier fix2col.sty, was written by
% Frank Mittelbach and David Carlisle. This package corrects a few problems
% in the LaTeX2e kernel, the most notable of which is that in current
% LaTeX2e releases, the ordering of single and double column floats is not
% guaranteed to be preserved. Thus, an unpatched LaTeX2e can allow a
% single column figure to be placed prior to an earlier double column
% figure.
% Be aware that LaTeX2e kernels dated 2015 and later have fixltx2e.sty's
% corrections already built into the system in which case a warning will
% be issued if an attempt is made to load fixltx2e.sty as it is no longer
% needed.
% The latest version and documentation can be found at:
% http://www.ctan.org/pkg/fixltx2e


%\usepackage{stfloats}
% stfloats.sty was written by Sigitas Tolusis. This package gives LaTeX2e
% the ability to do double column floats at the bottom of the page as well
% as the top. (e.g., "\begin{figure*}[!b]" is not normally possible in
% LaTeX2e). It also provides a command:
%\fnbelowfloat
% to enable the placement of footnotes below bottom floats (the standard
% LaTeX2e kernel puts them above bottom floats). This is an invasive package
% which rewrites many portions of the LaTeX2e float routines. It may not work
% with other packages that modify the LaTeX2e float routines. The latest
% version and documentation can be obtained at:
% http://www.ctan.org/pkg/stfloats
% Do not use the stfloats baselinefloat ability as the IEEE does not allow
% \baselineskip to stretch. Authors submitting work to the IEEE should note
% that the IEEE rarely uses double column equations and that authors should try
% to avoid such use. Do not be tempted to use the cuted.sty or midfloat.sty
% packages (also by Sigitas Tolusis) as the IEEE does not format its papers in
% such ways.
% Do not attempt to use stfloats with fixltx2e as they are incompatible.
% Instead, use Morten Hogholm'a dblfloatfix which combines the features
% of both fixltx2e and stfloats:
%
% \usepackage{dblfloatfix}
% The latest version can be found at:
% http://www.ctan.org/pkg/dblfloatfix




%\ifCLASSOPTIONcaptionsoff
%  \usepackage[nomarkers]{endfloat}
% \let\MYoriglatexcaption\caption
% \renewcommand{\caption}[2][\relax]{\MYoriglatexcaption[#2]{#2}}
%\fi
% endfloat.sty was written by James Darrell McCauley, Jeff Goldberg and
% Axel Sommerfeldt. This package may be useful when used in conjunction with
% IEEEtran.cls'  captionsoff option. Some IEEE journals/societies require that
% submissions have lists of figures/tables at the end of the paper and that
% figures/tables without any captions are placed on a page by themselves at
% the end of the document. If needed, the draftcls IEEEtran class option or
% \CLASSINPUTbaselinestretch interface can be used to increase the line
% spacing as well. Be sure and use the nomarkers option of endfloat to
% prevent endfloat from "marking" where the figures would have been placed
% in the text. The two hack lines of code above are a slight modification of
% that suggested by in the endfloat docs (section 8.4.1) to ensure that
% the full captions always appear in the list of figures/tables - even if
% the user used the short optional argument of \caption[]{}.
% IEEE papers do not typically make use of \caption[]'s optional argument,
% so this should not be an issue. A similar trick can be used to disable
% captions of packages such as subfig.sty that lack options to turn off
% the subcaptions:
% For subfig.sty:
% \let\MYorigsubfloat\subfloat
% \renewcommand{\subfloat}[2][\relax]{\MYorigsubfloat[]{#2}}
% However, the above trick will not work if both optional arguments of
% the \subfloat command are used. Furthermore, there needs to be a
% description of each subfigure *somewhere* and endfloat does not add
% subfigure captions to its list of figures. Thus, the best approach is to
% avoid the use of subfigure captions (many IEEE journals avoid them anyway)
% and instead reference/explain all the subfigures within the main caption.
% The latest version of endfloat.sty and its documentation can obtained at:
% http://www.ctan.org/pkg/endfloat
%
% The IEEEtran \ifCLASSOPTIONcaptionsoff conditional can also be used
% later in the document, say, to conditionally put the References on a
% page by themselves.




% *** PDF, URL AND HYPERLINK PACKAGES ***
%
%\usepackage{url}
% url.sty was written by Donald Arseneau. It provides better support for
% handling and breaking URLs. url.sty is already installed on most LaTeX
% systems. The latest version and documentation can be obtained at:
% http://www.ctan.org/pkg/url
% Basically, \url{my_url_here}.




% *** Do not adjust lengths that control margins, column widths, etc. ***
% *** Do not use packages that alter fonts (such as pslatex).         ***
% There should be no need to do such things with IEEEtran.cls V1.6 and later.
% (Unless specifically asked to do so by the journal or conference you plan
% to submit to, of course. )


% correct bad hyphenation here
\hyphenation{op-tical net-works semi-conduc-tor}

\begin{document}
%
% paper title
% Titles are generally capitalized except for words such as a, an, and, as,
% at, but, by, for, in, nor, of, on, or, the, to and up, which are usually
% not capitalized unless they are the first or last word of the title.
% Linebreaks \\ can be used within to get better formatting as desired.
% Do not put math or special symbols in the title.
\title{P8}
%
%
% author names and IEEE memberships
% note positions of commas and nonbreaking spaces ( ~ ) LaTeX will not break
% a structure at a ~ so this keeps an author's name from being broken across
% two lines.
% use \thanks{} to gain access to the first footnote area
% a separate \thanks must be used for each paragraph as LaTeX2e's \thanks
% was not built to handle multiple paragraphs
%

\author{S. R. Petersen, L. Svane, A. D. Nielsen, K. K. Malowanczyk, A. M. Landi% <-this % stops a space
}

% note the % following the last \IEEEmembership and also \thanks -
% these prevent an unwanted space from occurring between the last author name
% and the end of the author line. i.e., if you had this:
%
% \author{....lastname \thanks{...} \thanks{...} }
%                     ^------------^------------^----Do not want these spaces!
%
% a space would be appended to the last name and could cause every name on that
% line to be shifted left slightly. This is one of those "LaTeX things". For
% instance, "\textbf{A} \textbf{B}" will typeset as "A B" not "AB". To get
% "AB" then you have to do: "\textbf{A}\textbf{B}"
% \thanks is no different in this regard, so shield the last } of each \thanks
% that ends a line with a % and do not let a space in before the next \thanks.
% Spaces after \IEEEmembership other than the last one are OK (and needed) as
% you are supposed to have spaces between the names. For what it is worth,
% this is a minor point as most people would not even notice if the said evil
% space somehow managed to creep in.



% The paper headers
\markboth{P8 \today}%
{}
% The only time the second header will appear is for the odd numbered pages
% after the title page when using the twoside option.
%
% *** Note that you probably will NOT want to include the author's ***
% *** name in the headers of peer review papers.                   ***
% You can use \ifCLASSOPTIONpeerreview for conditional compilation here if
% you desire.




% If you want to put a publisher's ID mark on the page you can do it like
% this:
%\IEEEpubid{0000--0000/00\$00.00~\copyright~2015 IEEE}
% Remember, if you use this you must call \IEEEpubidadjcol in the second
% column for its text to clear the IEEEpubid mark.



% use for special paper notices
%\IEEEspecialpapernotice{(Invited Paper)}




% make the title area
\maketitle

% As a general rule, do not put math, special symbols or citations
% in the abstract or keywords.

%\begin{abstract}
%Model-based mutation testing is a method for automatically testing a system under test, and is considered a part of the model-based testing area of research. We propose a model transformation termed ``diabolic completion'' that allows for conformance testing directly in the UPPAAL tool. We have also developed a system to take a model, and perform diabolic completion, with the additions of allowing creation of mutants, conformance checking using the UPPAAL verification engine, and automatic test case generation. We then set up a case study using the car alarm system model commonly used in this area of research, which we tested MoMuT::TA, Ecdar 2.2, and our system on. In this case study we observed MoMuT::TA and Ecdar 2.2 to be at least a factor of two times slower than our system.
%\end{abstract}

\begin{abstract}
	The abstract should be placed here.
\end{abstract}

% Note that keywords are not normally used for peerreview papers.
%\begin{IEEEkeywords}

%mutation, testing, conformance testing, UPPAAL, timed automata, real-time systems, diabolic %completion, model-based mutation testing, model-based testing.
%\end{IEEEkeywords}






% For peer review papers, you can put extra information on the cover
% page as needed:
% \ifCLASSOPTIONpeerreview
% \begin{center} \bfseries EDICS Category: 3-BBND \end{center}
% \fi
%
% For peerreview papers, this IEEEtran command inserts a page break and
% creates the second title. It will be ignored for other modes.
\IEEEpeerreviewmaketitle

\section{Introduction}
  % An introduction of recommender systems, how they work and the problem we will address.
  Recommender systems are becoming more and more incorporated in our every day lives and are part of many different software systems \cite{recommender_e-comerce}. The benefit of a good recommender system can be many, and they range from increasing the sales numbers for a business and finding the correct product for a customer, to help researchers navigate the ever increasing knowledge base of scientific material. The boom in collected and available data through the widespread adaption of the internet has also made it possible to give better and more meaningful recommendations.
  % Find a paper for "commen for most recommeder systems ..."
  Common for most modern recommender systems is that they look at the node attributes of some data graph \cite{Ricci2015}. As an example, we can look at a webshop, where customers can buy products like books, clothes, furniture, etc. Here, there are different ways of recommending products to users. If a user $u$ searches for, or purchases a product $p$, a system can then find and recommend products with similar attributes to $p$. The system could also keep a list of items frequently bought together with product $p$ as an attribute of $p$, and recommend them to the user. This way of recommendation is formerly known as Content-based recommendation \ref{recommenders}. If the system also collects information about the users, it will be possible to find users that have attributes in common with $u$, find purchases of products that $u$ has not purchased, and recommend these. This approach is known as Collaborative filtering-based recommendation and may rely on node clustering for finding similar groups of users \ref{recommenders}. Common to all of these systems from a graph data perspective, is that they almost only look at node attributes for comparison and recommendation, and thus do not use the structural information of the graph, which is one of the benefits of this kind of data representation.

  % What we want to do in this paper:
  We would like to use different methods of comparing structural similarity of nodes in a graph and investigate, whether this approach can lead to better and more meaningful recommendations in recommender systems. Our main concern is to investigate the quality of recommendations achieved by using structural information, but we will also discuss the efficiency based on memory usage and runtime metrics as well as the scalability of the different methods. In particular, the methods we will discuss in this paper are simGNN, Type Extension Trees \ref{Related_work} and node/graph embedding with Node2Vec.
  % What we contribute with:
  The main contribution of this paper is thus an investigation and conclusion of the potential benefits from structural similarity methods in a recommender setting. The rest of this paper is structered as follows:
  % TODO: Add references to the definitions of content-based and collaborative filtering-based recommendation along with simGNN, TETs and embedding/Node2Vec.

\section{Related work}
\label{sec:Related_work}
In the article "Page Ranking Algorithms: A Survey" N. Duhan Et. Al.\cite{4809246} describes the PageRank algorithm related concept. The authors explain strengths and weaknesses with each stage the algorithm has been through. The original PageRank algorithm are easy to implement and is the basis of PageRank. Weighted PageRank can be more precise and runs faster by having a look at what input and output a page has.

"SimRank: A Measure of Structural-Context Similarity" written by G. Jeh and J. Widom\cite{10.1145/775047.775126} describes the SimRank algorithm and its uses on graphs. In SimRank two objects are similar if they are referenced by similar objects and an object is maximally similar to itself. This leads to similarity in different part of a graph to be hard to find and it uses vast amounts of memory $(O(n^2))$. Sometimes the ranking can seem illogical to a human, but SimRank is reliable in many cases of finding similar subgraphs.

"A Comprehensive Survey of Graph Embedding: Problems, Techniques, and Applications" by H. cai Et. Al.\cite{8294302} is a paper that summarizes and describes the terminology in graph embedding. They describe what an input graph can be and what kind of embedding its mostly used for. The embedding methods are plentiful and are often made to conserve data in compact graphs. The embedding method results in one of these type of embedded graphs, node-, edge-, subgraph- or whole graph embedding.

"Type Extension Trees for feature construction and learning in relational domains" by M. Jaeger Et. Al. \cite{JAEGER201330} defines framework called Type-Extension-Trees (TETs) which are used to represent complex counts-of-counts features which are complex combinatorial data structures in a relational domain. They present a learning algorithm which can be used to discover informative counts-of-counts features. These TET features can be made into prediction models either via the use of simple discriminant functions which allows for binary classification or via the use of a metric on TET values based on the Wasserstein-Kantorovich metric which allows for distance-based methods.

The article "Counts-of-counts similarity for prediction and search in relational data" by M. Jaeger Et. Al.\cite{jaeger2019counts}  gives a quick description of Type Extension Trees (TET) and Counts-of-Counts. The article uses the principles from "Type Extension Trees for feature construction and learning in relational domains" in forming the TETs and found substructures in databases. They showed that it is possible to calculate similarity using logical evaluation and histograms or by using earth mover's distance (EMD).

"Knowledge Graph Embedding for Link Prediction and Triplet Classification" an article by E. Shijia Et. Al. \cite{10.1007/978-981-10-3168-7_23} combats the problem of embedding knowledge graphs where graph embedding mostly has been used on social network graphs before.  Their solution is to learn semantic relations between entities in a form of edge embedding where they later would be able to ask logical questions and get one or more relations.

In "Semi-supervised Graph Embedding for Multi-label Graph Node Classification" by K. Gao Et. Al.\cite{10.1007/978-3-030-34223-4_35} they train a Graph convolution network to classify and embed subgraphs. They use a semi supervised technique because of the inconsistencies that can be found in multi-label graphs. This technique allows for unavailable relations to be represented, but since these relations are not part of the original graph, a "supervised learning only" strategy cannot be applied.

\section{Preliminaries}

\subsection{Recommenders}
\label{recommenders}
In order to solve some of the problems regarding recommender systems, we will now give a formal definition of recommender systems as described in computer science theory.

There are different types of recommender systems and in this paper we focus on the types that might work on sub graphs.

\textit{Content-based recommendation} is a recommendation system that analyses items preferred by a user to find common attributes of the items. The system then compares these attributes to attributes of other items to find similarities \cite{lu2015recommender}. Thus, Content-based recommendation is not dependent on other users but the attributes of the items.

\textit{Collaborative filtering-based recommendation} is a recommender system that represents users or items as vectors, such that a similarity value can be calculated between vectors. Collaborative filtering-based recommenders depend on data from other items or users, and these systems therefore give users trust in the answer but often have the cold start problem \cite{lu2015recommender}. In this context, trust is defined as when a system is able to represent why it recommended an item, and the cold start problem is a common problem in recommender systems where a database or a subset of a database has inadequate data, resulting in unreliable recommendations\cite{Ricci2015}.

\subsection{Graph Embedding}
In \cite{8294302} HongYun Cai et al. discuss the different ways of graph embedding. They define a taxonomy for the embedding strategies input and output.

The input taxonomy has four definitions.  

\begin{definition}[Homogeneous Graphs] A graph where both nodes and edges belong to a single type respectively, and can be further categorized by either adding weights, directions or both. In Directed and unweighted Graphs, distance between nodes dependent on how many edges they are related by. Two nodes that are related should be more similar than two where only one node is related to the other\cite{8294302}.
\end{definition}

\begin{definition}[Heterogeneous Graphs] A Heterogeneous Graph is a graph that can have nodes and edges of multiple types\cite{8294302}. 
\end{definition}

A commen example of Heterogeneous Graphs are Knowledge Graphs where edges and nodes usually represent different relations and entities respectively. For example, a knowledge graph over some film relations could have the entities "Director", "Actor" and "Film" represented as nodes, and the relations "Produced", "Directed" and "Acted-In" represented as edges.

\begin{definition}[Graph with Auxiliary Information] A graph that contains extra information for a node or relation. 
\end{definition}

An example could be a graph with the category of nodes called "Book", which contains information about the book and its author. The node in this case would be the name of the book and author could be auxiliary information\cite{8294302}.

\begin{definition}[Graph constructed from non-relational data] Instead of providing the input graph we construct it from non-relational input data, this is usually done when the input data is assumed to be in a low dimensional manifold. In this kind of graph relations can be discovered by using methods like K-Nearest-Neighbours\cite{8294302}.
\end{definition}

The taxonomy for output has four definitions as well.

\begin{definition}[Node embedding] The act of taking the nodes in the graph and represent them as a vector. When doing this two nodes that were similar in the graph should have similar vectors and therefore be able to be identified\cite{8294302}.
\end{definition}

\begin{definition}[Edge Embedding] Given a relation between two nodes a triplet is created \textit{<h,r,t>}  where h and t are the head and tail node and r is the relation using this it is possible to given two of the three components to predict what the last component should be\cite{8294302}.
\end{definition}

\begin{definition}[Hybrid Embedding] Spans over a few different embedding methods like substructure embedding where nodes and/or edges are embedded and then categorised to represent substructures of a graph. This can then be used to find substructures that are similar to each other\cite{8294302}.
\end{definition}

\begin{definition}[Whole-Graph Embedding] Representing the whole graph as one single vector and is therefore usually only done on small graphs like proteins and molecules\cite{8294302}. 
%But this is also the hardest one to do as an embedding method wants to be reversible, but this method can forget information because of the sheer amount of information required to be put together.
\end{definition}

To embed the graph there is several different techniques and each of them has different advantages and disadvantages.

\textit{Matrix Factorization} is a possible way to do graph embedding and is good because it considers the global proximity which will give the analyst a more precise estimation of one vector's distance to all other vectors. The method does have a disadvantage in that it scales linearly and can therefore be quite time consuming\cite{8294302}.

\textit{Deep Learning} is in many cases an effective method due to the relative speed of calculations and the small amount of memory space it uses. There is also some disadvantages for example in how the system trains. It is possible to train the system so it will not need any feature engineering but can also be hit by overfitting and underfitting which can cause large problems for the system\cite{8294302}.

\textit{Edge Reconstruction Based Optimization} is the term for three different methods "maximizing edge reconstruction, minimizing distance-based loss and minimizing margin-based loss" which in each there own ways ensure that the original graph input can be reconstructed from an embedded graph\cite{8294302}.

\textit{Graph Kernel} starts in a kernel and then walks through a subgraph comparing each kernel that was chosen. This is mostly used in graph embedding and only represent and compare structures that are desired to be compared. But this method have problems with substructures that are not independent and will grow the size exponentially\cite{8294302}.

All these embedding strategies all have applications and can be used for a lot of different classification, clustering and recommendation methods. \textit{Node Embedding} can be used for node classifications for SVM and KNN, node clustering in graphs and node recommendations by finding associations in the graph\cite{8294302}. 

\textit{Edge Embedding} is used for triplet classifications. As an example we can find if a relation $a-b$ goes through relation $r$ and finding which kind of link appears between two nodes in a graph. \textit{Hybrid and Whole Graph Embedding} will make it possible to classify graphs in a lower dimension than they originally where and this embedding method is better for visualization of the graph for human consumption\cite{8294302}.

\subsection{SimRank}
An popular algorithm for computing the similarity between substructures in graphs is SimRank, created by Jeh et. al\cite{10.1145/775047.775126}. The intuition behind the SimRank algorithm is as the author states in the paper \emph{"two objects are similar if they are referenced by similar objects"}\cite{10.1145/775047.775126}. The algorithm assigns a similarity score between 0 and 1 based on how similar two objects are. The basic SimRank equation can be seen below:
\begin{definition}[SimRank]\label{def:simrank} Given two objects $a$ and $b$ as input we compute the similarity $s(a,b) \in [0,1]$ via the following recursive equation:
	\begin{equation}\label{eq: basic_simrank}
	s(a,b)= \frac{C}{|I(a)||I(b)|}\sum^{|I(a)|}_{i=1}\sum^{|I(b)|}_{j=1}s(I_i(a),I_j(b))
	\end{equation}
	Here C is the confidence level or decay factor which is a constant between 0 and 1. If $a=\emptyset$ and $b= \emptyset$ then $s(a,b) = 0$. If both $a$ and $b$ are not $\emptyset$ then we compute $s(a,b)$ by iterating over all in-neighbor pairs of $a$ and $b$ and sum up the similarity $s(I_i(a),I_j(b)$ of these pairs. The similarity is then divided by the total number of in-neighbor pairs in order to normalize it. That is, the similarity between a and b is the average similarity between in-neighbors of $a$ and in-neighbors of $b$. It is important to note here that the similarity between an object and itself is 1\cite{10.1145/775047.775126}.
\end{definition}

%\Jeg er lidt usikker på om det er nødvendigt at snakke om bipartite simRank så det kan godt være at alt dette ikke er nødvendigt
It is also possible to extend the above equation to bipartite domains consisting of two types of objects. The idea here is that we can compute similarity of two objects of one type based the objects they reference which are of another type. An example of this could be trying to figure out the similarity two customers based on the items they have bought\cite{10.1145/775047.775126}.The equations for computing the similarity of object in bipartite domains can be seen below:

\begin{definition}[Bipartite SimRank]\label{def:bipartite_simrank} Given two objects A and B of type 1 which reference objects of type 2 and objects c and of type 2 which gets referenced by objects of type 1 the similarity can be computed via the follow mutually-recursive equations:
	
	For objects of type 1 where $A \neq B$ the similarity is computed as:
	\begin{equation}
	s(A,B)= \frac{C_1}{|O(A)||O(B)|}\sum^{|O(A)|}_{i=1}\sum^{|O(B)|}_{j=1}s(O_i(A),O_j(B))
	\end{equation}
	
	For objects of type 2 where $c \neq d$ the similarity is computed as:
	\begin{equation}
	s(c,d)= \frac{C_2}{I|(c)||I(d)|}\sum^{|I(c)|}_{i=1}\sum^{|I(d)|}_{j=1}s(I_i(c),I_j(d))
	\end{equation}
	
	We can then compute the bipartite SimRank score by checking $C_1 = C_2$\cite{10.1145/775047.775126}.
\end{definition}

The way that SimRank is computed for a graph $G$ can be done via iterating to fixed-point. Let $n$ be the number of nodes within graph $G$, for each iteration $k$ we can keep $n^2$ entries $R_k(*,*)$ of length $n^2$, where $R_k(a,b)$ give the similarity score between $a$ and $b$ on iteration $k$\cite{10.1145/775047.775126}. We can compute $R_{k+1}(a,b)$ based on $R_k(a,b)$. First we have to compute $R_0(a,b)$ which is the lower bound of the similarity score $s(a,b)$:
\begin{equation}
R_0(a,b)= \begin{cases}
0, & (\text{if } a \neq b) \\

1 \quad (\text{if } a = b)
\end{cases}
\end{equation}

To compute compute $R_{k+1}(a,b)$ from $R_k(*,*)$ we use the equation below which is simply a modification of the basic SimRank equation mentioned earlier:
\begin{equation}
R_{k+1}(a,b)= \frac{C}{|I(a)||I(b)|}\sum^{|I(a)|}_{i=1}\sum^{|I(b)|}_{j=1}s(I_i(a),I_j(b))
\end{equation}

What the above equation states is that for $a \neq b$ and $R_{k+1}(a,b) = 1$ for $a = b$ we update for $(a,b)$ on each iteration $k+1$ using the similarity of the neighbors of $(a,b)$ from the previous iteration $k$. The values $R_k(*,*)$ are nondecreasing as $k$ increases. 

\subsection{Structural Similarity}
\label{Subsec:structural_similarity}
Structural similarity is a widely used term that is used in many different areas, including graph comparison. The way that we define structural similarity is, that when looking at two graphs or subgraphs we look at the nodes from one graph and comparing them with nodes from another graph. The similarity between the nodes is done by checking the type of adjacent nodes connected to the currently observed nodes. 
A node $n$ and a node $n'$ are similar if they have a similar set of adjacent nodes connected to them.

\subsection{SimGNN}\label{AP:SimGNN}
One of the methods that we tried to use was Similarity Computation via Graph Neural Networks or SimGNN which is an algorithm used for calculating the similarity between two graphs created by Bai et al. \cite{Bai2018}. The algorithm was intended to be utilized to compare two users from the Movielens dataset but was found to be too complex when it comes to finding Graph Edit Distance (GED). This was discovered upon running through the test set after SimGNN\footnote{The original code we used for computing SimGNN can be found here: https://github.com/benedekrozemberczki/SimGNN} had been trained where the mean squared error results turned out to be 0.023. While these result are in themselves are very good it made us suspicion that the result might be too good. After further investigation we concluded that our graphs which SimGNN trains on are supposed to predict are very simple and that SimGNN therefore isn't suited for our use case. Combine this with the fact that SimGNN take about 2 days to train on our dataset and that the GED for two graphs from our dataset can be computed with a simpler and faster implementation only reinforces this notion. The remainder of this section is a explanation on how SimGNN works.

The algorithm works by using two different strategies. The first primary strategy is to compute the similarity between two graphs based on interaction between their graph level embeddings. The second auxiliary strategy computes similarity based on two sets of node-level embeddings. The second strategy is optional and only helps to give a more accurate prediction at the cost of runtime\cite{Bai2018}.

Both strategies in SimGNN require that the graphs get embedded at the node level. To perform this embedding Bai et. al. uses a Graph Convolutional Network (GCN)\cite{Bai2018} which is created by Kipf et. al. \cite{Kipf2016}. The main operation behind GCN in SimGNN is shown in \autoref{Eq:GCN}.


\begin{equation}\label{Eq:GCN}
conv(u_n)=f_1(\sum_{m \in N(n)} \frac{1}{\sqrt{d_nd_m}}u_mW_1^{(l)}+b_1^{(l)})
\end{equation}

In \autoref{Eq:GCN} $u_n$ is a representation of a node. $N(n)$ is the set of first-order neighbors of a node $n$ plus $n$ itself, $d_n$ is the degree of node $n$ plus 1, $w_1^{(l)}$ is a weight matrix associated l-th GCN layer, $b_1^{(l)}$ is the bias and $f_1()$ is an activation function. The idea of \autoref{Eq:GCN} is that it aggregates features from the first-order neighbor of a node $n$\cite{Bai2018}.

What is happening inside GCN is the following. Given a graph $G$ the GCN f(X,A) takes as input: a feature matrix $X$ and an adjacency matrix $A$ which represents G. Inside each hidden layer in the GCN the following propagation takes place\cite{Kipf2016}:



\begin{equation}
\label{eq:propagation_rule}
H^{l+1} = \sigma (\tilde{D}^{-0.5}*\tilde{A}*\tilde{D}^{0.5}*H{(l)} * W^{(l)})
\end{equation}

In \autoref{eq:propagation_rule} $\tilde{A} = A +I_N$ is the adjacency matrix of the graph with an added self-loop which is the identity matrix $I_N$. $\tilde{D} = \sum_j * \tilde{A}_ij$ is the degree matrix which is computed by summing up the feature representation of the neighbors of $\tilde{A}$. The reason why $\tilde{D}$ is to the power of $^{-0.5}$ is to avoid vanishing/exploding gradients. $H^{(l)}$ is the matrix of activations from the previous l'th layer with $H^0$ representing the input layer, thereby being the feature matrix X. $W^{(l)}$ is the weigh matrix from the previous l'th layer and lastly $\sigma$ represents an activation function such as ReLU\cite{Kipf2016}. The nodes from \autoref{eq:propagation_rule} ends up being represented as the sum of the features from the neighbor node and the current node itself multiplied with a weight and then applying an activation function. 


Once node-level embedding has been computed, the first strategy can use these to compute the graph-level embeddings. Here Bai et. al. present an attention mechanism that is used to figure out which nodes are more important than others with respect to a specific similarity metric. Consequently more important nodes should therefore receive higher weights\cite{Bai2018}. The equation for the attention mechanism can be seen \autoref{Eq:Att}.



\begin{equation}\label{Eq:Att}
h= \sum^N_{n=1} \sigma(u^T_nc)u_n=\sum^N_{n=1}\sigma(u^T_ntanh((\frac{1}{N}\sum_{m=1}^Nu_n)W_2))u_n
\end{equation}

The idea behind \autoref{Eq:Att} is that when creating a graph embedding $h$ we compute a global graph context $c$ which is done by taking the average of all the node-level embedding and feeding them into a nonlinear transformation function. This means that the global context $c=tanh((\frac{1}{N}\sum_{m=1}^Nu_m)W_2)$ where $W_2$ is a learning matrix. Once $c$ has been computed we can compute the attention weight for each node $n$. The idea is that nodes which are similar to the global context should receive higher attention weights. To ensure that the attention weights are in a range of 0 and 1 a sigmoid activation function $\sigma$ is being used\cite{Bai2018}.


The graph-level embeddings are then fed into a Neural Tensor Network to model their relation with each other, seen in \autoref{Eq:NTN}.


\begin{equation}\label{Eq:NTN}
g(h_i,h_j) = f_3(h_i^TW_3^{[1:K]}h_j+V\begin{bmatrix}h_i \\ h_j \end{bmatrix} + b_3)
\end{equation}

In \autoref{Eq:NTN} $h_i$ and $h_j$ are the two graph-level embeddings we would like to compute the relation on. $W_3^{[1:K]}$ is a weight tensor, $b_3$ is a bias,  $f_3$ is an activation function and $K$ is a hyperparameter used for controlling the number of similarity scores produced by the model for each graph-level embedding pair\cite{Bai2018}. This then produces a list of similarity scores which are then feed into a standard feedforward neural network to reduce the dimensionality of the similarity score and finally only produce a single similarity scores which is the predicted graph edit distance between the two graphs. This prediction is the compared with the expected result using the mean squared error loss function\cite{Bai2018}.


Besides computing a coarse comparison between the two graphs which is what was done in the first strategy, SimGNN also allows for supplementing with a finer node level comparison which is what the second strategy does.

The second strategy utilizes the same node-level embedding computed by the GCN algorithm as the first strategy a6nd then a set of pairwise interaction score are computed by $S = \sigma(U_iU_j^T)$ where $U_i$ and $U_j$ are node embedding for the two graphs. $S$ is also called a pairwise similarity matrix. In the case where one graph has less nodes than the other a set of fakes node are padded on. Once the $S$ has been computed we extract histogram features $hist(S)$. These histogram features are then normalized and concatenated with the graph-level interaction scores computed in the first strategy and then fed into a feed-forward neural network\cite{Bai2018}.




\section{Implementation}

\subsection{TET Histograms and Distance}
	Comparing the TET can be done in different ways. One of the method used to find similarity between the TETs is to transform the TET into a histogram is representing the by using the vector representation from \autoref{Eq:TETvector} we can calculate each substructures percentile distribution. this would give a representation as \autoref{fig:Tethistogram}
	
	\begin{figure}[H]
		\centering
		\begin{adjustbox}{width=0.5\textwidth}
			\input{Article/figures/TET_histogram_figure.tex}
		\end{adjustbox}
		\caption{caption}
		\label{fig:Tethistogram}
	\end{figure}
	
	By using this representation as a vector we can compare two histograms with simple distance measures as forekamle manhatten or euclidean distance.
	
	When comparing two histograms they will most likely have some elements the other does not have the histogram without the element will simply get 0 for the missing value and thereby be able to add distance for the missing element. A simple example done with manhattan distance can be seen in \autoref{Eq:manhattencomparason}.
	
	\begin{equation}\label{Eq:manhattencomparason}
	D(\begin{bmatrix}
	x_{1.1} \\
	x_{1.2} \\
	\end{bmatrix},
	\begin{bmatrix}
	x_{2.1} \\
	x_{2.2} \\
	x_{2.3}
	\end{bmatrix})= |x_{1.1} - x_{2.1}| + |x_{1.2} - x_{2.2}| + |0 - x_{2.1}|
	\end{equation}
	
	This is easily implemented if the vectors are programmed as dictionaries by going through, the union of keywords in the two dictionaries and returning 0 it a key is not found.


\subsection{Graph Edit Distance}\label{Subsec:GED}
As the Graph Edit Distance metric is used in the SimGNN algorithm mentioned in \autoref{AP:SimGNN}, we give a quick explanation of here before continuing with the rest of the appendix.

Graph Edit Distance (GED) is a metric for comparing graphs. The GED of two graphs $a$ and $b$ is a numeric distance value between the graphs. As seen in \autoref{Eq:ged}, the GED is calculated as the fewest number of edits needed in order to transform one graph into the other\cite{Riesen2015}.

\begin{equation}\label{Eq:ged}
 GED(g,g') = \min_{(e_1,e_2 \dots e_k) \in P(g, g') } \sum_{i=1}^k c(e_i)
\end{equation}

In \autoref{Eq:ged} $P(g,g')$ is the set of paths being edited and $c$ is the cost function, measuring the strength of a node edit operation $e_i$\cite{Riesen2015}.

\subsection{Metric Tree representation}
  Since we can assume, with good reason, that datasets used in recommender systems only will increase in size, we are interested in minimizing the computation time of structural data queries, like the k-nearest neighbor query.
  In order to find the exact k-nearest neighbors for only one user, we would have to compare the user with all the other users.
  This will result in a linear runtime, which is usually categorized as efficient in the theory of algorithm design and implementation. But since the datasets used in recommender systems tend to be very large, including millions of users, having to calculate the distance between one user and millions of other users may become quite time consuming.
  Furthermore, if each user has to be compared to any other user, a brute force k-nearest neighbor approach will lead to a quadratic runtime.
  As a solution to this problem, we can use the Metric Tree (MT) datastructure for improving nearest neighbor retrieval.
  We refer to \cite{jaeger2019counts} and \cite{uhlmann1991} for the complete definition of MTs, but we give a short explanation of the parts relevant to this paper.

  The Metric Tree is a datastructure that consists of two types of nodes, internal nodes and leaf nodes. An internal node contains two entities and two branches. A leaf node contains a set of entities, also known as the bucket.
  A MT is constructed by a recursive procedure in which a dataset is split. The procedure splits data and recursively calls itself over each of the subsets, until a stopping condition is met.
  If the maximal tree depth is reached, or the current set to be splitted is not larger than the maximal bucket size, a leaf node is returned.
  If not, two entities $z1$ and $z2$ are chosen randomly from the dataset, and data are split by their distances to these entities.

  With the Metric trees implemented, we can use a greedy algorithm for finding the approximate k-nearest-neighbors for a user.
  As stated in \cite{jaeger2019counts}, given a metric tree, the fastest solution for approximate k-nearest-neighbor retrieval for a query object is to traverse the tree, following at each node the branch whose corresponding entity is closer to the query one, until a leaf node is found.
  We can then sort the entities in the bucket contained in the leaf node according to their distance to the query object, and return the k nearest neighbors. We note here that this only is an approximate solution, but it will provide us with results that are more than accurate for recommendation tasks.


\section{Experiment}\label{Experiment}
In this section we will describe how we setup our experiment in subsection \ref{setup}. We then explain in subsection \ref{Sec:Reliability} how we wish to prove reliability. Lastly we describe our results in subsection \ref{results}.

\subsection{Setup}\label{Subsec:setup}
In this experiment we test the reliability of the TET compared to the embedding and a bruteforce methods. 

For our experiments we use the MovieLens dataset provided by grouplens\cite{Grouplensdata}.
The data is taken from MovieLens, a movie recommendation website. The dataset we use contains $100000$ movie ratings, given by $943$ users, applied to $1682$ movies.
All users have rated at least $20$ movies and each user and movie is represented by an Id.
The data was generated between September 19th, 1997 and April 22nd, 1998. For the users no additional information is available, other than what can be inferred from the rating data. For the movies we have additional information such as Movie Title, Release year, and Genres.

\begin{figure}[H]
	\centering
	\begin{adjustbox}{width=0.5\textwidth}
		\input{Article/figures/graph_representation.tex}
	\end{adjustbox}
	\caption{The simple graph}
	\label{fig:graph_representation}
\end{figure}

We create a weighted graph from the data, represented as an edgelist.
Each edge is of the form (MovieId, UserId, Rating) where the rating acts as a weight.
Each node in the network is either a User or a Movie represented by their Id with the addition of a $U$ on userid and $M$ on movieid respectively, to differentiate them.
The movie node will contain the additional data of genre internally.
A simple example of the graph can be seen in \autoref{fig:graph_representation}.

The data will be split in 10 folds as shown on \autoref{fig:tenfold} using a $80\%/20\%$ split into training- and test set \cite{Ricci2015}.
We would like to also add a validation set and the split will therefore be $80\%/10\%/10\%$ or 8 folds/1 fold/1 fold.
The splits will be on edge level meaning that training a user will have $80\%$ of their ratings.
We will in the validation and test reconstruct or predict the ratings in the set.

\begin{figure}[H]
	\centering
	\begin{adjustbox}{width=0.5\textwidth}
		\input{Article/figures/tenfold.tex}
	\end{adjustbox}
	\caption{10 fold example}
	\label{fig:tenfold}
\end{figure}

These split will then be used for building the TETs and embedding nodes.

When these models have been build and trained we will use them to find k-nearest user neighbors according  to each model.
With these neighbors we are able to make a prediction on the rating between movies and a user.
For the predictions we will use each training sets edge list by using the rating we know to calculate potential edges. 
We use \autoref{eq:pred} to calculate a predicted rating for the missing ratings in the adjacency matrix.

\begin{equation}\label{eq:pred}
pred(u,m,N) = \overline{r_u}+\sum_{n \in N}w(u,n, N)(r_{n,m}-\overline{r_n})
\end{equation}

\begin{equation}\label{eq:W}
w(u,n, N)=sim(u,n)/\sum_{n' \in N} sim(u,n')
\end{equation}

In \autoref{eq:pred} $u$ and $m$ is the user and movie to calculate a rating between, $N$ is the set of nearest neighbors.
Nearness is in this account defined by sim(u,n) the simultaneity between users u and n.
$\overline{r_a}$ is the average rating for user $a$ and $r_{b,m}$ is the rating given by user $b$ to movie $m$.
Additionally in \autoref{eq:W} $n$ is a neighboring user to $u$.

The ratings that are missing should be found in the set of the validation fold and test folds.

We will use root mean square error as seen in  \autoref{eq:RMSE}\cite{chai2014root} to find the accuracy of the methods using different folds for test, training and validation set.
Where $y_i \in Y$ is the predicted ratings and $x_i \in X$ is the known ratings from the test or validation set. Each rating independent of set is in the set ${1,2,3,4,5}$.

\begin{equation}\label{eq:RMSE}
RMSE = \sqrt{\frac{1}{n}\sum^n_{i=1}(y_i - x_i)^2}
\end{equation}

A low RMSE is better than a higher RMSE, and a low RMSE is equivalent to a high accuracy.

% There are subtleties that we need to take care of and explain.
	% Do we devide the data on the node level or the edge level?
	% This determines what would we be testing?
	% Lars: i would like to divide on edge level and test on the ability to construct or reconstruct edges between movies and users. We could do the same with list and try to reconstruct the list of bedst movies for a user
% Leave-one-out method
% Splitting the data into folds
	% Lars: i think we should make 20 folds this would allow us to take the standart 70/30 model for splitting training and test and addabt it to a 70/15/15 for training, validation and test.
% Recommender Systems handbook. Read the chapter on evaulating recommender systems.
% Before the experiment: 
	% what is the **Hypothesis**?
	% what is our **Variables**?
	% can the conclutiuon be **Generalized**?
% Offline experiments:
	% what is the **dataset**?
	% what kind of **behavior** does it simulate?

\subsection{Reliability}\label{Subsec:Reliability}
We will now recall some theory on Reliability and use it to evaluate the reliability of our solution. In order to prove Reliability, we can construct several hypotheses that together ensure a notion of reliability for our system. In the theory on
Experiments and Evaluation, hypotheses are categorized into two categories called \textit{one sample hypothesis} and \textit{two sample hypothesis}. In this case, a one sample hypothesis makes a statement about the system performance, whereas a two sample hypothesis makes a comparison. To answer each kind of hypothesis, two corresponding categories of tests exist, called \textit{one sample test} and \textit{two sample test}.
In order to ensure reliability for our system, we will setup some one sample hypotheses regarding the Runtime and Correctness of our system. We will emphasize the Correctness part, this will tell if the system provides users with correct and reliable recommendations.
In order to compare our different approaches and solutions, we will also setup some two sample hypothesis regarding the before mentioned parameters. 
We can measure Runtime as a cost, whereas Correctness is categorized as the gain from the systems.

Since we use the baseline as our comparison for the methods we will be using this for the two sample hypothesis, which we will compare each method to. 
We have the hypothesis that the correctness of each method, will be better than the baseline method, because of this does not hold up then the method would not be useful.
Another hypothesis is that the runtime of the TET and node2vec methods will be running faster than a standard brute force method.

%To test our methods correctness we decided to implement a Root Mean Squared Error (RMSE) value and then compare this value against the baseline method. If a method were not able to create a better prediction than the baseline, then the method would be taken as not reliable.
%We also tested the runtime of each algorithm, but had not set a certain value on what would make it unreliable, but was used more as a description for how well the algorithm would work in a real use case.

%As seen in the result section \autoref{results}
%In this paper we used Root Mean Squared Error to calculate our correctness, this was done on each of the methods introduced individually and then it was compared with a baseline which was the average rating of the fold used as the prediction for the entire fold. This was done for each fold through all the 10 fold cross validation sets. We had several hypothesis about the methods that we wanted to check for, for example we wanted each model to be better at predicting than the baseline. If a model were not able to predict better than a baseline, then the method would not be feasible for this use case and therefore not be realiable.


%\subsection{Reliability}\label{reliability}
%To make a reliable recommendation we first have to discuss what we mean when we say reliability. Reliability in this paper means if a recommender can within a reasonable amount of time for the use case, precure a accurate result then the recommender system would be taken as reliable. Accuracy will be measured with Root Mean Squared Error (RMSE) on all methods used and then these RMSE will be compared to a baseline that is set by taking the average rating from each fold, given by a 10 fold cross validation, and this average being given as the prediction in all cases. For each of the methods introduced we make the hypotheses that they will work for finding structural similarity in graphs for use in recommender systems. To test if these hypothesis are correct we will be comparing them to the baseline method and if they are worse than the baseline, then they will not be useful for recommendation. After we have completed our results we will then evaulate if the hypothesis still hold up.

%We will now recall some theory on Reliability and use it to evaluate the reliability of our solution. In order to prove Reliability, we can construct several hypotheses that together ensure a notion of reliability for our system. In the theory on
%Experiments and Evaluation, hypotheses are categorized into two categories called \textit{one sample hypothesis} and \textit{two sample hypothesis}. In this case, a one sample hypothesis makes a statement about the system performance, whereas a two sample hypothesis makes a comparison. To answer each kind of hypothesis, two corresponding categories of tests exist, called \textit{one sample test} and \textit{two sample test}.
%In order to ensure reliability for our system, we will setup some one sample hypotheses regarding the Runtime, Memory usage, and Correctness of our system. We will emphasize the Correctness part, since this will tell us, if the system provides users with correct and reliable recommendations. In order to compare our different approaches and solutions, we will also setup some two sample hypothesis regarding the before mentioned parameters. Overall, we can measure Runtime and Memory usage as a cost, whereas Correctness can be categorized as the gain from the systems. 

\subsection{Result analysis}

\begin{figure}[H]
	\centering
	\begin{adjustbox}{width=0.5\textwidth}
		\input{Article/figures/base_10_fold_error.tex}
	\end{adjustbox}
	\caption{base error from 100k dating dataset }
	\label{fig:base_errors}
\end{figure}

\begin{figure}[H]
	\centering
	\begin{adjustbox}{width=0.5\textwidth}
		\input{Article/figures/brute_10_fold_error.tex}
	\end{adjustbox}
	\caption{bruteforce error from 100k dating dataset }
	\label{fig:brute_errors}
\end{figure}

\begin{figure}[H]
	\centering
	\begin{adjustbox}{width=0.5\textwidth}
		\input{Article/figures/N2V_10_fold_error.tex}
	\end{adjustbox}
	\caption{Node2vec error from 100k dating dataset }
	\label{fig:N2V_errors}
\end{figure}


Each of the test had an average run time of 1 hour and 12 minutes.  Fold 6 had the fasted run time of about 48 minutes but this fold might have a fitting problem as seen by the gap between validation and test error.

\begin{figure}[H]
	\centering
	\begin{adjustbox}{width=0.5\textwidth}
		\input{Article/figures/tet_10_fold_error.tex}
	\end{adjustbox}
	\caption{TET error from 100k dating dataset }
	\label{fig:tet_errors}
\end{figure}

In the graph of relative errors seen on \autoref{fig:tet_errors} we see that most folds lie at about $3.75$ error and that the fold with the lowest validation error has one of the biggest gap to test error this indicates a fitting mistake. 

Fold 5 and Fold 9 both have some interesting properties fold 9 has the same error for validation and test witch indicates a more general fold. Fold 5 has less error and thereby more accuracy and with and error difference between validation of 0.01 this might be the most reliable fold.


%\begin{figure}[H]
%	\centering
%	\begin{adjustbox}{width=0.5\textwidth}
%		\input{Article/figures/SimGNN_10_fold_error.tex}
%	\end{adjustbox}
%	\caption{TET error from 100k dating dataset }
%	\label{fig:tet_errors}
%\end{figure}

\subsection{Reliability}\label{Subsec:Reliability}
We will now recall some theory on Reliability and use it to evaluate the reliability of our solution. In order to prove Reliability, we can construct several hypotheses that together ensure a notion of reliability for our system. In the theory on
Experiments and Evaluation, hypotheses are categorized into two categories called \textit{one sample hypothesis} and \textit{two sample hypothesis}. In this case, a one sample hypothesis makes a statement about the system performance, whereas a two sample hypothesis makes a comparison. To answer each kind of hypothesis, two corresponding categories of tests exist, called \textit{one sample test} and \textit{two sample test}.
In order to ensure reliability for our system, we will setup some one sample hypotheses regarding the Runtime and Correctness of our system. We will emphasize the Correctness part, this will tell if the system provides users with correct and reliable recommendations.
In order to compare our different approaches and solutions, we will also setup some two sample hypothesis regarding the before mentioned parameters. 
We can measure Runtime as a cost, whereas Correctness is categorized as the gain from the systems.

Since we use the baseline as our comparison for the methods we will be using this for the two sample hypothesis, which we will compare each method to. 
We have the hypothesis that the correctness of each method, will be better than the baseline method, because of this does not hold up then the method would not be useful.
Another hypothesis is that the runtime of the TET and node2vec methods will be running faster than a standard brute force method.

%To test our methods correctness we decided to implement a Root Mean Squared Error (RMSE) value and then compare this value against the baseline method. If a method were not able to create a better prediction than the baseline, then the method would be taken as not reliable.
%We also tested the runtime of each algorithm, but had not set a certain value on what would make it unreliable, but was used more as a description for how well the algorithm would work in a real use case.

%As seen in the result section \autoref{results}
%In this paper we used Root Mean Squared Error to calculate our correctness, this was done on each of the methods introduced individually and then it was compared with a baseline which was the average rating of the fold used as the prediction for the entire fold. This was done for each fold through all the 10 fold cross validation sets. We had several hypothesis about the methods that we wanted to check for, for example we wanted each model to be better at predicting than the baseline. If a model were not able to predict better than a baseline, then the method would not be feasible for this use case and therefore not be realiable.


%\subsection{Reliability}\label{reliability}
%To make a reliable recommendation we first have to discuss what we mean when we say reliability. Reliability in this paper means if a recommender can within a reasonable amount of time for the use case, precure a accurate result then the recommender system would be taken as reliable. Accuracy will be measured with Root Mean Squared Error (RMSE) on all methods used and then these RMSE will be compared to a baseline that is set by taking the average rating from each fold, given by a 10 fold cross validation, and this average being given as the prediction in all cases. For each of the methods introduced we make the hypotheses that they will work for finding structural similarity in graphs for use in recommender systems. To test if these hypothesis are correct we will be comparing them to the baseline method and if they are worse than the baseline, then they will not be useful for recommendation. After we have completed our results we will then evaulate if the hypothesis still hold up.

%We will now recall some theory on Reliability and use it to evaluate the reliability of our solution. In order to prove Reliability, we can construct several hypotheses that together ensure a notion of reliability for our system. In the theory on
%Experiments and Evaluation, hypotheses are categorized into two categories called \textit{one sample hypothesis} and \textit{two sample hypothesis}. In this case, a one sample hypothesis makes a statement about the system performance, whereas a two sample hypothesis makes a comparison. To answer each kind of hypothesis, two corresponding categories of tests exist, called \textit{one sample test} and \textit{two sample test}.
%In order to ensure reliability for our system, we will setup some one sample hypotheses regarding the Runtime, Memory usage, and Correctness of our system. We will emphasize the Correctness part, since this will tell us, if the system provides users with correct and reliable recommendations. In order to compare our different approaches and solutions, we will also setup some two sample hypothesis regarding the before mentioned parameters. Overall, we can measure Runtime and Memory usage as a cost, whereas Correctness can be categorized as the gain from the systems. 


% An example of a floating figure using the graphicx package.
% Note that \label must occur AFTER (or within) \caption.
% For figures, \caption should occur after the \includegraphics.
% Note that IEEEtran v1.7 and later has special internal code that
% is designed to preserve the operation of \label within \caption
% even when the captionsoff option is in effect. However, because
% of issues like this, it may be the safest practice to put all your
% \label just after \caption rather than within \caption{}.
%
% Reminder: the "draftcls" or "draftclsnofoot", not "draft", class
% option should be used if it is desired that the figures are to be
% displayed while in draft mode.
%
%\begin{figure}[!t]
%\centering
%\includegraphics[width=2.5in]{myfigure}
% where an .eps filename suffix will be assumed under latex,
% and a .pdf suffix will be assumed for pdflatex; or what has been declared
% via \DeclareGraphicsExtensions.
%\caption{Simulation results for the network.}
%\label{fig_sim}
%\end{figure}

% Note that the IEEE typically puts floats only at the top, even when this
% results in a large percentage of a column being occupied by floats.


% An example of a double column floating figure using two subfigures.
% (The subfig.sty package must be loaded for this to work.)
% The subfigure \label commands are set within each subfloat command,
% and the \label for the overall figure must come after \caption.
% \hfil is used as a separator to get equal spacing.
% Watch out that the combined width of all the subfigures on a
% line do not exceed the text width or a line break will occur.
%
%\begin{figure*}[!t]
%\centering
%\subfloat[Case I]{\includegraphics[width=2.5in]{box}%
%\label{fig_first_case}}
%\hfil
%\subfloat[Case II]{\includegraphics[width=2.5in]{box}%
%\label{fig_second_case}}
%\caption{Simulation results for the network.}
%\label{fig_sim}
%\end{figure*}
%
% Note that often IEEE papers with subfigures do not employ subfigure
% captions (using the optional argument to \subfloat[]), but instead will
% reference/describe all of them (a), (b), etc., within the main caption.
% Be aware that for subfig.sty to generate the (a), (b), etc., subfigure
% labels, the optional argument to \subfloat must be present. If a
% subcaption is not desired, just leave its contents blank,
% e.g., \subfloat[].


% An example of a floating table. Note that, for IEEE style tables, the
% \caption command should come BEFORE the table and, given that table
% captions serve much like titles, are usually capitalized except for words
% such as a, an, and, as, at, but, by, for, in, nor, of, on, or, the, to
% and up, which are usually not capitalized unless they are the first or
% last word of the caption. Table text will default to \footnotesize as
% the IEEE normally uses this smaller font for tables.
% The \label must come after \caption as always.
%
%\begin{table}[!t]
%% increase table row spacing, adjust to taste
%\renewcommand{\arraystretch}{1.3}
% if using array.sty, it might be a good idea to tweak the value of
% \extrarowheight as needed to properly center the text within the cells
%\caption{An Example of a Table}
%\label{table_example}
%\centering
%% Some packages, such as MDW tools, offer better commands for making tables
%% than the plain LaTeX2e tabular which is used here.
%\begin{tabular}{|c||c|}
%\hline
%One & Two\\
%\hline
%Three & Four\\
%\hline
%\end{tabular}
%\end{table}


% Note that the IEEE does not put floats in the very first column
% - or typically anywhere on the first page for that matter. Also,
% in-text middle ("here") positioning is typically not used, but it
% is allowed and encouraged for Computer Society conferences (but
% not Computer Society journals). Most IEEE journals/conferences use
% top floats exclusively.
% Note that, LaTeX2e, unlike IEEE journals/conferences, places
% footnotes above bottom floats. This can be corrected via the
% \fnbelowfloat command of the stfloats package.








% if have a single appendix:
%\appendix[Proof of the Zonklar Equations]
% or
%\appendix  % for no appendix heading
% do not use \section anymore after \appendix, only \section*
% is possibly needed

% use appendices with more than one appendix
% then use \section to start each appendix
% you must declare a \section before using any
% \subsection or using \label (\appendices by itself
% starts a section numbered zero.)
%


% you can choose not to have a title for an appendix
% if you want by leaving the argument blank





% Can use something like this to put references on a page
% by themselves when using endfloat and the captionsoff option.
\ifCLASSOPTIONcaptionsoff
  \newpage
\fi



% trigger a \newpage just before the given reference
% number - used to balance the columns on the last page
% adjust value as needed - may need to be readjusted if
% the document is modified later
%\IEEEtriggeratref{8}
% The "triggered" command can be changed if desired:
%\IEEEtriggercmd{\enlargethispage{-5in}}

% references section

% can use a bibliography generated by BibTeX as a .bbl file
% BibTeX documentation can be easily obtained at:
% http://mirror.ctan.org/biblio/bibtex/contrib/doc/
% The IEEEtran BibTeX style support page is at:
% http://www.michaelshell.org/tex/ieeetran/bibtex/
%\bibliographystyle{IEEEtran}
% argument is your BibTeX string definitions and bibliography database(s)
%\bibliography{IEEEabrv,../bib/paper}
%
% <OR> manually copy in the resultant .bbl file
% set second argument of \begin to the number of references
% (used to reserve space for the reference number labels box)

%\begin{thebibliography}{1}

%\end{thebibliography}
\bibliographystyle{IEEEtran}
\bibliography{bib}





% biography section
%
% If you have an EPS/PDF photo (graphicx package needed) extra braces are
% needed around the contents of the optional argument to biography to prevent
% the LaTeX parser from getting confused when it sees the complicated
% \includegraphics command within an optional argument. (You could create
% your own custom macro containing the \includegraphics command to make things
% simpler here.)
%\begin{IEEEbiography}[{\includegraphics[width=1in,height=1.25in,clip,keepaspectratio]{mshell}}]{Michael Shell}
% or if you just want to reserve a space for a photo:


% insert where needed to balance the two columns on the last page with
% biographies
%\newpage


% You can push biographies down or up by placing
% a \vfill before or after them. The appropriate
% use of \vfill depends on what kind of text is
% on the last page and whether or not the columns
% are being equalized.

%\vfill

% Can be used to pull up biographies so that the bottom of the last one
% is flush with the other column.
%\enlargethispage{-5in}



% that's all folks
\end{document}
