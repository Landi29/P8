\begin{definition}[Homogeneous Graphs] A graph where both nodes and edges belong to a single type respectively, and can be further categorized by either adding weights, directions or both.\cite{8294302}.
\end{definition}

\begin{definition}[Heterogeneous Graphs] A Heterogeneous Graph is a graph that can have nodes and edges of multiple types\cite{8294302}.
\end{definition}

A common example of Heterogeneous Graphs are Knowledge Graphs where edges and nodes usually represent different relations and entities respectively. For example, a knowledge graph over some film relations could have the entities "Director", "Actor" and "Film" represented as nodes, and the relations "Produced", "Directed" and "Acted-In" represented as edges.

\begin{definition}[Graph with Auxiliary Information] A graph that contains extra information for a node or relation.
\end{definition}

An example could be a graph with the category of nodes called "Book", which contains information about the book and its author. The node in this case would be the name of the book and author could be auxiliary information\cite{8294302}.

\begin{definition}[Graph constructed from non-relational data] Instead of providing the input graph we construct it from non-relational input data, this is usually done when the input data is assumed to be in a low dimensional manifold. In this kind of graph relations can be discovered by using methods like K-Nearest-Neighbours\cite{8294302}.
\end{definition}
