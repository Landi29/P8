\subsection{Graph Embedding}
\label{Subsec:Graph_embedding}
An important factor when working with graphs is creating a representation of the graphs that can be worked on efficiently. Once such method is graph embedding which transforms graphs into vectors such that they can more effectively be worked on. In this section we will describe the underlying theory of graph embedding.
In \cite{8294302} HongYun Cai et al. discuss the different ways of graph embedding. They define a taxonomy for the embedding strategies input and output.

The input taxonomy has four definitions, consisting of four different graph types. The taxonomy for output has four definitions as well, describing the different embedding strategies. The first kind of embedded output is \textit{Node Embedding}. Here the nodes in the graph are represented as a vector. When doing Node Embedding, two nodes that were similar in the graph should have similar vectors and therefore be able to be identified\cite{8294302}. The second kind of embedding, is \textit{Edge Embedding}. Given a relation between two nodes, a triplet $<h,r,t>$ is created,  where $h$ and $t$ are the head and tail node, and $r$ is the relation. Using Edge Embedding, it is possible, given two of the three components, to predict what the last component should be\cite{8294302}.


It is then a relevant question, if Node and Edge embedding can be combined. The result is \textit{Hybrid Embedding}, which spans over a few different embedding methods. One of them is \textit{Substructure Embedding} where nodes and/or edges are embedded and then categorised to represent substructures of a graph. This can then be used to find substructures that are similar to each other\cite{8294302}.
Adding another layer of abstraction to \textit{Substructure Embedding} we get \textit{Whole-Graph Embedding} which represents the whole graph as one single vector and is therefore usually only done on small graphs like proteins and molecules\cite{8294302}.

To embed a graph there are several different techniques and each of them has different advantages and disadvantages.

%\textit{Matrix Factorization} is a simple embedding model and is good because it considers the global proximity which will give the analyst a more precise estimation of one vector's distance to all other vectors. The method does have a disadvantage in that it scales linearly and can therefore be quite time consuming for large data sets\cite{8294302}.

\textit{Deep Learning} is in many cases an effective method due to the relative speed of calculations and the small amount of memory space it uses. There are also some disadvantages for example in how the system trains. It is possible to train the system so it will not need any feature engineering but can also be hit by overfitting and underfitting which can cause large problems for the system\cite{8294302}.

\textit{Edge Reconstruction Based Optimization} is the term for three different methods: maximizing edge reconstruction, minimizing distance-based loss and minimizing margin-based loss. All of which in there own ways ensure that the original graph input can be reconstructed from an embedded graph\cite{8294302}.

\textit{Graph Kernel} starts in a kernel and then walks through a subgraph comparing each kernel that was chosen. This is mostly used in graph embedding and only represent and compare structures that are desired to be compared. But this method have problems with substructures that are not independent and will grow the size exponentially\cite{8294302}.

All these embedding strategies all have applications and can be used for a lot of different classification, clustering and recommendation methods. \textit{Node Embedding} can be used for node classifications for SVM and KNN, node clustering in graphs and node recommendations by finding associations in the graph\cite{8294302}.

\textit{Edge Embedding} is used for triplet classifications. As an example we can find if a relation $a-b$ goes through relation $r$ and finding which kind of link appears between two nodes in a graph. \textit{Hybrid and Whole Graph Embedding} will make it possible to classify graphs in a lower dimension than they originally where and this embedding method is better for visualization of the graph for human consumption\cite{8294302}.

