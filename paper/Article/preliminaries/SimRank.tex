\subsection{SimRank}
\label{Subsec:SimRank}
A popular algorithm for computing the similarity between substructures in graphs is SimRank, created by Jeh et al.\cite{10.1145/775047.775126}.
The intuition behind the SimRank algorithm is as the author states in the paper \emph{"two objects are similar if they are referenced by similar objects"}\cite{10.1145/775047.775126}.
The algorithm assigns a similarity score between 0 and 1 based on how similar two objects are. The basic SimRank equation can be seen below:

\begin{definition}[SimRank]\label{def:simrank} Given two objects $a$ and $b$ as input we compute the similarity $s(a,b) \in [0,1]$ via the following recursive equation:
	\begin{equation}\label{eq:basic_simrank}
	s(a,b)= \frac{C}{|I(a)||I(b)|}\sum^{|I(a)|}_{i=1}\sum^{|I(b)|}_{j=1}s(I_i(a),I_j(b))
	\end{equation}
\end{definition}

In \autoref{def:simrank} for SimRank, $C$ is the confidence level or decay factor which is a constant between $0$ and $1$.
If either $a$ or $b$ have no in-neighbours the similarity would be $s(a,b) = 0$.
If the neighbors of $a$ and $b$  is not $\emptyset$ then we can compute their similarity $s(a,b)$ by iterating through all neighbor pairs between $a$ and $b$.
While iterating we can sum up all the similarities of each pair and then divide it by the total number of neighbors to normalize the value.
This means that the similarity between two nodes $a$ and $b$ is found by taking the average similarity of neighbors of these points.
An object is always fully similar with itself so $s(a,a) = 1$.


%\Jeg er lidt usikker på om det er nødvendigt at snakke om bipartite simRank så det kan godt være at alt dette ikke er nødvendigt
Equation \ref{eq:basic_simrank} can be extended into the bipartite domains where objects can be one of two types for example users and items.
In this domain we can compute similarity of two objects of a single type based on the objects referenced by the other type.
So we could for example say that the similarity of two users can be computed based on the items they purchase.
The equations for computing the similarity of objects in bipartite domains are defined in \autoref{def:bipartite_simrank}.


%\Problemet her er jeg ikke ved hvad du bruger C til. I din formel er det en value mellem 0 - 1 men du bruger det som et nyt objekt. Hvad var tanken?
\begin{definition}[Bipartite SimRank]\label{def:bipartite_simrank} Let s(A,B) be the similarity between two objects A and B of type 1 and for nodes of type 2 we denote it as s(M1,M2) where M1 and M2 are of type 2. $C_1$ and $C_2$ are both constants between 0 and 1.

	For objects of type 1 where $A \neq B$ the similarity is computed as:
	\begin{equation}\label{eq:bipartite_simrank1}
	s(A,B)= \frac{C_1}{|O(A)||O(B)|}\sum^{|O(A)|}_{i=1}\sum^{|O(B)|}_{j=1}s(O_i(A),O_j(B))
	\end{equation}

	For objects of type 2 where $M1 \neq M2$ the similarity is computed as:

	\begin{equation}\label{eq:bipartite_simrank2}
	s(M1, M2)= \frac{C_2}{I|(M1)||I(M2)|}\sum^{|I(M1)|}_{i=1}\sum^{|I(M2)|}_{j=1}s(I_i(M1),I_j(M2))
	\end{equation}

	We can then compute the bipartite SimRank score by checking $C_1 = C_2$\cite{10.1145/775047.775126}.
\end{definition}

The way that SimRank is computed for a graph $G = (V,E)$ can be done via iterating to fixed-point.
Let $n = |V|$ be the number of nodes within graph $G$, for each iteration $k$ we can keep $n^2$ entries $R_k(*,*)$ of length $n^2$.
$R_k(a,b)$ where $\{a,b\} \in V$ gives the similarity score between $a$ and $b$ on iteration $k$\cite{10.1145/775047.775126}.
We can compute $R_{k+1}(a,b)$ based on $R_k(a,b)$.
First we have to compute $R_0(a,b)$ which is the lower bound of the similarity score $s(a,b)$:

\begin{equation}\label{eq:lowerbound_sim_score}
R_0(a,b)= \begin{cases}
0, & (\text{if } a \neq b) \\

1 ,& (\text{if } a = b)
\end{cases}
\end{equation}

To compute $R_{k+1}(a,b)$ from $R_k(*,*)$ we use the equation below which is simply a modification of the basic SimRank from \autoref{eq:basic_simrank}:
\begin{equation}\label{eq:simrank_computation}
R_{k+1}(a,b)= \frac{C}{|I(a)||I(b)|}\sum^{|I(a)|}_{i=1}\sum^{|I(b)|}_{j=1}s(I_i(a),I_j(b))
\end{equation}

What \autoref{eq:simrank_computation} states is that for $a \neq b$ and $R_{k+1}(a,b) = 1$ for $a = b$ we update for $(a,b)$ on each iteration $k+1$ using the similarity of the neighbors of $(a,b)$ from the previous iteration $k$. The values $R_k(*,*)$ are nondecreasing as $k$ increases.
