\subsection{Recommenders}
In order to solve some of the problems regarding recommender systems, we will now give a formal definition of recommender systems as described in computer science theory. 

There are different types of recommender systems and in this paper we focus on the types that might work on sub graphs.

\begin{definition}[Content-based recommendation] \label{def:Content_based_recommendation}
A recommendation system that analyses items preferred by a user to find common attributes of the items, and compares these attributes to attributes of other items to find similarities \cite{lu2015recommender}. 
\end{definition}

Content-based recommendation as defined in definition  \autoref{def:Content_based_recommendation} is not dependent on other users but the attributes of the items.

\begin{definition}[Collaborative filtering-based recommendation]\label{def:Collaborative_filtering_based}
A recommender system that represents users or items as vectors, such that a simulation can be calculated between vectors\cite{lu2015recommender}.
\end{definition}

Collaborative filtering-based recommenders defined in definition \autoref{def:Collaborative_filtering_based}, depend on data from other items or users. These systems therefore give users trust in the answer but often have the cold start problem.

\begin{definition}[Trust]\label{def:trust}
When a recommendation is reliable due to the systems ability to represent why it recommended an item\cite{Ricci2015}.
\end{definition}

\begin{definition}[Cold Start Problem]\label{def:cold_start_problem}
A common problem in recommender system when the database or a subset of the database has inadequate data, resulting in unreliable recommendations\cite{Ricci2015}.
\end{definition}
