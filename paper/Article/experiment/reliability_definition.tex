\subsection{Reliability}\label{Subsec:Reliability_definition}

In order to evaluate the results of our solution, we would like to investigate it from a reliability perspective.
Reliability can be many things depending on the science and research area it is used for, and we will thus start by giving a definition of Reliability in regards to Recommender systems

For a recommender to be \textit{reliable}, it must be \textit{accurate}, \textit{fast} and \textit{trustworthy}.
By accurate, we mean that the recommender predicts rating estimates that are close to the ratings, the user would actually give the recommended items.
A recommender system must also be idempotent, thus producing the same results under similar conditions. Furthermore, a system must take frequent updates into account and be able to manage them, as new data is constantly provided.
For a recommender system to be reliable, it must also be robust, thus being able to withstand attempts of manipulation. As a result of this, the system should only collect the amount of data necessary, in order to give accurate recommendations.

We will focus on accuracy as the main measure and speed as a secondary measure for Reliability, because we are not analyzing a full recommender system, but instead focus on the task of recommendation.
We measure accuracy as an average precision value, and as a consequence, we cannot say anything about best or worst case scenarios. This can be done with the \textit{Root Mean Square Error (RMSE)} metric. Using RMSE and defining a baseline value leads us to the following hypothesis:

\begin{hypothesis}\label{RMSE_hypothesis}
  Brute force based recommendation, recommendation with node embedding and recommendation with TETs will perform better than the baseline RMSE.
\end{hypothesis}

As the secondary measure, speed will be measured in seconds per person's recommendations and leads us to the hypothesis:
\begin{hypothesis}\label{Speed_hypothesis}
  The average recommendation time is less than 10 seconds.
\end{hypothesis}

We measure the accuracy with RMSE because it is one of two standard measures for measuring the difference between predicted and observed values. The other measure called \textit{Mean Absolute Error} would also work to some degree, but RMSE is less tolerant towards off values and thus better at telling us, if we have predicted wrong values.
