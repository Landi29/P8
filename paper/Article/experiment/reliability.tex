\subsection{Reliability}\label{Subsec:Reliability}
We will now recall some theory on Reliability and use it to evaluate the reliability of our solution. In order to prove Reliability, we can construct several hypotheses that together ensure a notion of reliability for our system. In the theory on
Experiments and Evaluation, hypotheses are categorized into two categories called \textit{one sample hypothesis} and \textit{two sample hypothesis}. In this case, a one sample hypothesis makes a statement about the system performance, whereas a two sample hypothesis makes a comparison. To answer each kind of hypothesis, two corresponding categories of tests exist, called \textit{one sample test} and \textit{two sample test}.
In order to ensure reliability for our system, we will setup some one sample hypotheses regarding the Runtime and Correctness of our system. We will emphasize the Correctness part, this will tell if the system provides users with correct and reliable recommendations.
In order to compare our different approaches and solutions, we will also setup some two sample hypothesis regarding the before mentioned parameters. 
We can measure Runtime as a cost, whereas Correctness is categorized as the gain from the systems.

Since we use the baseline as our comparison for the methods we will be using this for the two sample hypothesis, which we will compare each method to. 
We have the hypothesis that the correctness of each method, will be better than the baseline method, because of this does not hold up then the method would not be useful.
Another hypothesis is that the runtime of the TET and node2vec methods will be running faster than a standard brute force method.

%To test our methods correctness we decided to implement a Root Mean Squared Error (RMSE) value and then compare this value against the baseline method. If a method were not able to create a better prediction than the baseline, then the method would be taken as not reliable.
%We also tested the runtime of each algorithm, but had not set a certain value on what would make it unreliable, but was used more as a description for how well the algorithm would work in a real use case.

%As seen in the result section \autoref{results}
%In this paper we used Root Mean Squared Error to calculate our correctness, this was done on each of the methods introduced individually and then it was compared with a baseline which was the average rating of the fold used as the prediction for the entire fold. This was done for each fold through all the 10 fold cross validation sets. We had several hypothesis about the methods that we wanted to check for, for example we wanted each model to be better at predicting than the baseline. If a model were not able to predict better than a baseline, then the method would not be feasible for this use case and therefore not be realiable.


%\subsection{Reliability}\label{reliability}
%To make a reliable recommendation we first have to discuss what we mean when we say reliability. Reliability in this paper means if a recommender can within a reasonable amount of time for the use case, precure a accurate result then the recommender system would be taken as reliable. Accuracy will be measured with Root Mean Squared Error (RMSE) on all methods used and then these RMSE will be compared to a baseline that is set by taking the average rating from each fold, given by a 10 fold cross validation, and this average being given as the prediction in all cases. For each of the methods introduced we make the hypotheses that they will work for finding structural similarity in graphs for use in recommender systems. To test if these hypothesis are correct we will be comparing them to the baseline method and if they are worse than the baseline, then they will not be useful for recommendation. After we have completed our results we will then evaulate if the hypothesis still hold up.

%We will now recall some theory on Reliability and use it to evaluate the reliability of our solution. In order to prove Reliability, we can construct several hypotheses that together ensure a notion of reliability for our system. In the theory on
%Experiments and Evaluation, hypotheses are categorized into two categories called \textit{one sample hypothesis} and \textit{two sample hypothesis}. In this case, a one sample hypothesis makes a statement about the system performance, whereas a two sample hypothesis makes a comparison. To answer each kind of hypothesis, two corresponding categories of tests exist, called \textit{one sample test} and \textit{two sample test}.
%In order to ensure reliability for our system, we will setup some one sample hypotheses regarding the Runtime, Memory usage, and Correctness of our system. We will emphasize the Correctness part, since this will tell us, if the system provides users with correct and reliable recommendations. In order to compare our different approaches and solutions, we will also setup some two sample hypothesis regarding the before mentioned parameters. Overall, we can measure Runtime and Memory usage as a cost, whereas Correctness can be categorized as the gain from the systems. 
