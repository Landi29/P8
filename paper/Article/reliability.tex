\subsection{Reliability}
We will now recall some theory on Reliability and use it to evaluate the reliability of our solution. In order to prove Reliability, we can construct several hypotheses that together ensure a notion of reliability for our system. In the theory on
Experiments and Evaluation, hypotheses are categorized into two categories called \textit{one sample hypothesis} and \textit{two sample hypothesis}. In this case, a one sample hypothesis makes a statement about the system performance, whereas a two sample hypothesis makes a comparison. To answer each kind of hypothesis, two corresponding categories of tests exist, called \textit{one sample test} and \textit{two sample test}.
In order to ensure reliability for our system, we will setup some one sample hypotheses regarding the Runtime, Memory usage, and Correctness of our system. We will emphasize the Correctness part, since this will tell us, if the system provides users with correct and reliable recommendations. In order to compare our different approaches and solutions, we will also setup some two sample hypothesis regarding the before mentioned parameters. Overall, we can measure Runtime and Memory usage as a cost, whereas Correctness can be categorized as the gain from the systems.
