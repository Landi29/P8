\subsection{SimRank}
\label{Subsec:SimRank}
We looked into using SimRank as a way to find the similarity of nodes in a graph network, using the data described in \autoref{Subsec:setup}, where we have an undirected graph with users' movie ratings.
SimRank is a popular algorithm for computing the similarity between substructures in graphs, and is created by Jeh et al.\cite{10.1145/775047.775126}.
The intuition behind the SimRank algorithm is that \emph{"two objects are similar if they are referenced by similar objects"}\cite{10.1145/775047.775126}. The algorithm assigns a similarity score between 0 and 1 based on how similar two objects are.

The idea was to use SimRank as a way to find structual equivalence between the users in a graph, based on the similarity score, and to recommend movies to them like it is done with the other methods explained in \autoref{Subsec:setup}.

The way SimRank works is by computing the similarities for all nodes in a graph $G = (V,E)$ by iterating to a fixed-point.
Let $n = |V|$ be the number of nodes within graph $G$, for each iteration $k$ we keep $n^2$ entries $R_k(*,*)$.
$R_k(a,b)$ where $\{a,b\} \in V$, gives the similarity score between $a$ and $b$ on iteration $k$\cite{10.1145/775047.775126}.
We compute $R_{k+1}(a,b)$ based on prior iteration $R_k(a,b)$.
First we compute $R_0(a,b)$ which is the lower bound of the similarity score.

\begin{definition}[SimRank]\label{def:simrank} Given two objects $a$ and $b$ as input, we compute the similarity $R_{k+1}(a,b) \in [0,1]$ based on the prior similarities $R_k(*,*)$.

For iteration $R_0$ the similarity scores are initialized by the equation:
\begin{equation}\label{eq:lowerbound_sim_score}
R_0(a,b)= \begin{cases}
0, & (\text{if } a \neq b) \\

1 ,& (\text{if } a = b)
\end{cases}
\end{equation}

To compute $R_{k+1}(a,b)$ from $R_k(*,*)$ we use the equation:
\begin{equation}\label{eq:simrank_computation}
R_{k+1}(a,b)= \frac{C}{|I(a)||I(b)|}\sum^{|I(a)|}_{i=1}\sum^{|I(b)|}_{j=1}s(I_i(a),I_j(b))
\end{equation}

$C$ is the confidence level or decay factor which is a constant between $0$ and $1$, $I(n)$ is the in-neighbors for a node $n \in V$.

\end{definition}

What \autoref{eq:simrank_computation} states is that for $a \neq b$ and $R_{k+1}(a,b) = 1$ for $a = b$, we update the similarity score $(a,b)$ on each iteration $k+1$ using the similarity of the in-neighbors of $(a,b)$ from the previous iteration $k$. The values of $R_k(*,*)$ are non-decreasing as $k$ increases.
We compute the similarity $R_k(a,b)$ by iterating through all neighbor pairs between $a$ and $b$, summing up all the similarities of each pair and then dividing it by the total number of neighbors to normalize the value.
This means that the similarity between two nodes $a$ and $b$ is found by taking the average similarity of neighbors of these points. An object is always fully similar with itself so $s(a,a) = 1$.

Using this method we tested SimRank on our graph network from \autoref{Subsec:setup}. This was done using NetworkX, which implements this method. The NetworkX implementation includes an option to choose the number of iterations as well as a tolerance threshold. That checks for each similarity score if the change in similarity between the new iteration $R_{k+1}$ and the prior iteration $R_k$ is below the threshold and if so, stops the algorithm accordingly.

Using the implementation on our graph, the algorithm would continue to run for a very long time, making it unfeasible to use for our case.
The method was also tested on a few different graphs created using $NetworkX.cycle_graph(n)$ of various sizes, up to $2500$ number nodes, creating a graph with cyclic connections. All were able to run and terminate within ten minutes.
The difference in time between our graph network and the ones created with NetworkX could be due to the vast difference in the number of neighbors that each node has.
In the Movielens dataset with 100 thousand ratings, each node has an average of $225,64$ neighbors. 
This makes the graph more connected which means that the number of in-neighbors and number of similarity scores to calculate for each call is much higher.
Given the results we concluded that the current implementation is not feasible for the type of graph network we are using and a change in the network structure or SimRank implementation is needed.
