\section{Conclusion}\label{Sec:Conclusion}
Throughout this article we have discussed the intricacies of node2vec and Type Extension Trees, and we have found that both these frameworks will make it possible to give recommendations that are more precise than an average rating baseline.
Node2vec performs better when there is information on the particular product that you wish to recommend a person.
This method uses products and users nearness in a network, but will have problems recommending outside the proximity and context of the user.
TETs on the other hand recommends best with the total power of the data set and does not need the nearness of a product to make a recommendation.
Under normal circumstances node2vec were able to make recommendations with an average RMSE over the dataset at about 0.95 in a 1 to 5 rating system, compared to the TETs' at about 1 RMSE.
These results were both better than the baseline, thereby fulfilling the hypotheses.
In an item cold start scenario the TETs performed better with an average RMSE of 0.80 in a 1 to 3 rating system where node2vec ended with an average RMSE at 0.95.
In this scenario, we were again able to get better RMSEs than the baseline.
With all the models we tested  the prediction of edges for all products in the validation and test sets to their respective user.
As each test finished in under 10 second, it shows that the implementations we used are feasible for the recommendation task .
As the results of the accuracy and speed tests were able to prove the hypotheses made about reliability, we can conclude that the methods are reliable.
We also note that the methods have both advantages and disadvantages depending on the use case.

\subsection{Future Work}
A future task for this project could be to extend the TETs and try other features, making the TETs more expressible. This could give the TETs better recommendations.
Another way that could improve the usability of the product, could be to inspect the scalability of the algorithms.
By doing this we might be able to decrease the runtime of the TETs in their training phase and reduce the amount of time it takes to perform a recommendation in general.
It could also be a possibility to test out different parameters using node2vec, changing the number of dimensions, walks and the p and q parameter, to see if it is possible to make a more precise model for recommendations.
Another task is to look into ways to make graph embedding more scalable so that it can be performed on larger and more connected graphs within a decent time frame and with feasible memory usage.
We would like to test our used methods on more complex graphs testing if the methods is scalable in that way.
