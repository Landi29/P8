\section{Introduction}
  % An introduction of recommender systems, how they work and the problem we will address.
  Recommender systems are becoming more and more incorporated in our every day lives and are part of many different software systems. The benefit of a good recommender system can be many, and they range from increasing the sales numbers for a business and finding the correct product for a customer, to help researchers navigate the ever increasing knowledge base of scientific material. The boom in collected and available data through the widespread adaption of the internet has also made it possible to give better and more meaningful recommendations. Common for most modern recommender systems is that they look at the node attributes of some data graph. As an example, we can look at a webshop, where customers can buy products like books, clothes, furniture, etc. Here, there are different ways of recommending products to users. If a user $u$ searches for, or purchases a product $p$, a system can then find and recommend products with similar attributes to $p$. The system could also keep a list of items frequently both together with product $p$ as an attribute of $p$, and recommend them to the user. If the system collects information about the users, it will also be possible to find users with common attributes to $u$, find purchases of product that $u$ has not purchased, and recommend these. Common to all of these systems from a graph data perspective, is that they almost only look at node attributes for comparison and recommendation, and thus do not use the structural information of the graph, which is one of the benefits of this kind of data representation.
  %To specify, what theory and which challenges we will work with, we formulate the problem statement below:
  %\begin{center}
    %\textit{How can we define and use structural similarity on graphs for recommendation tasks?}
  %\end{center}

  %Furthermore, we will also discuss the following:
  %\begin{itemize}
  %  \item What current Machine Learning methods can help define and compute structural similarity between nodes in a graph?
  %  \item Which sets of features for nodes and subgraphs will generate good recommendations?
  %  \item What relevant theory, data structures and metrics are available for discovering and processing structural similarity on graphs?
  %  \item How can graph based similarity methods help solve some of the challenges with regards to recommender systems?
  %\end{itemize}
