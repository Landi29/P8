\section{Introduction}
  Recommender systems are becoming more and more incorporated in our every day lives and are part of many different software systems \cite{recommender_e-comerce}. The benefit of a good recommender system can be many,  ranging from increasing the sales numbers for a business by finding the correct product for a customer, to helping researchers navigate the ever increasing knowledge base of scientific material. The increase in collected and available data through the widespread adaption of the internet has also made it possible to give better and more meaningful recommendations.

  Common for most modern recommender systems is that they look at the node attributes of some data graph \cite{Ricci2015}. As an example, we can look at a webshop where there are several different ways of recommending products to users. If a user $u$ searches for, or purchases a product $p$, a system can then find and recommend products with attributes similar to $p$. This way of recommendation is formally known as Content-based recommendation.

  If the system also collects information about the users, it will be possible to find users that are close to $u$ in the graph, and recommend products they have purchased. This approach is known as Collaborative filtering-based recommendation and may rely on node clustering for finding similar groups of users. Based on the same approach, many systems also implement a \textit{frequently bought with} feature, where the system recommends other items purchased along with product $p$.

  Given the increase in the amount of data produced through the internet, some valuable recommendations might be missed by either of the two methods, since a proper abstraction is missing. We would like to use different methods of comparing structural similarity of nodes in a graph and investigate, whether this approach can lead to better and more meaningful recommendations in recommender systems. Our main concern is to investigate the quality of recommendations achieved by using structural information, but we will also discuss the efficiency based on memory usage and runtime metrics as well as the scalability of the different methods. In particular, the methods we will discuss in this paper are Type Extension Trees and node embedding with Node2Vec.

  The main contribution of this paper is thus an investigation and conclusion of the potential benefits from structural similarity methods in a recommendation setting. The rest of this paper is structured as follows:
  In section \ref{sec:Related_work} we will discuss various pieces of work related to our study. Section \ref{Sec:preliminaries} will delve into describing the theory of behind all the methods we wish to study along with theory of the recommendation area of study. Section \ref{Sec:Implementation} and \ref{Sec:Experiment} will describe our practical implementation of the methods and how our experiment went respectively. 
  \todo{Finish off this section by writing about discussion, conclussion, future work and potentially also appendix}
