\section{Introduction}
  % An introduction of recommender systems, how they work and the problem we will address.
  Recommender systems are becoming more and more incorporated in our every day lives and are part of many different software systems \cite{recommender_e-comerce}. The benefit of a good recommender system can be many, and they range from increasing the sales numbers for a business and finding the correct product for a customer, to help researchers navigate the ever increasing knowledge base of scientific material. The boom in collected and available data through the widespread adaption of the internet has also made it possible to give better and more meaningful recommendations.
  % Find a paper for "commen for most recommeder systems ..."
  Common for most modern recommender systems is that they look at the node attributes of some data graph \cite{Ricci2015}. As an example, we can look at a webshop, where customers can buy products like books, clothes, furniture, etc. Here, there are different ways of recommending products to users. If a user $u$ searches for, or purchases a product $p$, a system can then find and recommend products with similar attributes to $p$. The system could also keep a list of items frequently bought together with product $p$ as an attribute of $p$, and recommend them to the user. This way of recommendation is formerly known as Content-based recommendation \ref{recommenders}. If the system also collects information about the users, it will be possible to find users that have attributes in common with $u$, find purchases of products that $u$ has not purchased, and recommend these. This approach is known as Collaborative filtering-based recommendation and may rely on node clustering for finding similar groups of users \ref{recommenders}. Common to all of these systems from a graph data perspective, is that they almost only look at node attributes for comparison and recommendation, and thus do not use the structural information of the graph, which is one of the benefits of this kind of data representation.

  % What we want to do in this paper:
  We would like to use different methods of comparing structural similarity of nodes in a graph and investigate, whether this approach can lead to better and more meaningful recommendations in recommender systems. Our main concern is to investigate the quality of recommendations achieved by using structural information, but we will also discuss the efficiency based on memory usage and runtime metrics as well as the scalability of the different methods. In particular, the methods we will discuss in this paper are simGNN, Type Extension Trees \ref{Related_work} and node/graph embedding with Node2Vec.
  % What we contribute with:
  The main contribution of this paper is thus an investigation and conclusion of the potential benefits from structural similarity methods in a recommender setting. The rest of this paper is structered as follows:
  % TODO: Add references to the definitions of content-based and collaborative filtering-based recommendation along with simGNN, TETs and embedding/Node2Vec.
