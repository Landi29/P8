\subsection{Node2vec}

As mentioned earlier in \ref{Sec:node2vec_prelim} we will use node2vec as a method to embed nodes in a network in a vector space. In the node2vec paper, Aditya Grover and Jure Leskovec provides a python implementation of node2vec. An updated python3 version with some performance improvements, by eliorc (github citation) is available, and will be the implementation used as other implementations are also done in python.

The implementation by eliorc makes use of the NetworkX package to represent the network that you want to embed. For creating the network, we first discretize the data into an edgelist representation. Let $E$ be the set of all edges and $e = \{node, node, weight\}$ be the representation for an edge. For each $e \in E$ add e as an edge to our Networkx graph, this gives us both our edges and nodes in a graph object $G$.

The Node2vec implementation have set of parameters that it takes aswell as the graph. $Node2vec(G, d, wl, nw, p, q)$ where $G$ is the graph, $d$ is the number of dimensions, $wl$ is the walk length, $nw$ is the number of walks for each node, $p$ and $q$ is for tuning the sampling strategy towards a bias.

This gives us a list of all random walks

run the gensim.word2veec by fitting the model, with parameters $node2vec.fit(window, min_count, batch_words)$

as output we get an object containing all the embeddings that were learned.

Explain result with similarity, cosine similarity