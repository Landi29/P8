\subsection{Metric Tree representation}
  Since we can assume, with good reason, that datasets used in recommender systems only will increase in size, we are interested in minimizing the computation time of structural data queries, like the k-nearest neighbor query. In order to find the exact k-nearest neighbors for only one user, we would have to compare the user with all the other users. This will result in a linear runtime, which is usually categorized as efficient in the theory of algorithm design and implementation. But since the datasets used in recommender systems tend to be very large, including millions of users, having to calculate the distance between one user and millions of other users may become quite time consuming. Furthermore, if each user has to be compared to any other user, this brute force k-nearest neighbor approach will lead to a quadratic runtime. As a solution to this problem, we can use the Metric tree datastructure for improving nearest neighbor retrieval.
  % Explain briefly how we build Metric trees and search them.
  % Explain how it can improve the performance of the implementation.
